\documentclass{sig-alternate}
\usepackage[hyphens]{url}
\begin{document}
\nocite{*}

\conferenceinfo{DSCE}{'14 Evanston, Illinois USA}

\title{Cliq: Something Something}
% \subtitle{EECS 395/495: Distributed Systems in Challenging Environments}

\numberofauthors{3}

\author{
\alignauthor Angela Jiang\\
  \affaddr{Northwestern University}\\
  \email{\small angelajiang2014@u.northwestern.edu}
\alignauthor Andrew Kahn\\
  \affaddr{Northwestern University}\\
  \email{\small andrewkahn2015@u.northwestern.edu}
\and
\alignauthor Max New\\
  \affaddr{Northwestern University}\\
  \email{\small maxnew2014@u.northwestern.edu}
}

\date{9 June 2014}

\maketitle
\begin{abstract}
A social news site presents user-curated content, ranked by popularity. Popular 
curators like Reddit, or Facebook have become effective way of crowdsourcing 
news or sharing personal opinions. Traditionally, these services require a 
centralized authority to aggregate data and determine what to display. However, 
the trust issues that arise from a centralized system are particularly damaging to 
the "Web democracy" that social news sites are meant to provide. 

In this poster, we present {\bf cliq}, a decentralized social news curator. cliq is a P2P based 
social news curator that provides private and unbiased reporting. All users in cliq share 
responsibility for tracking and providing popular content. Any user data that cliq needs 
to store is also managed across the network. We first inform our design of cliq through 
an analysis of Reddit. We design a way to provide content curation without a persistent 
moderator, or usernames.
\end{abstract}

\begin{section}{Introduction}
Unlike other media, content shown on social media sites is determined by a "Web democracy", 
open to anyone. However, the trust issues that arise from a centralized system are particularly 
damaging to this goal. Users must accept that owners of these sites have the power to present 
content aligned with their own agenda. For example, BBC news recently reported that moderators 
of the subreddit "technology" were censoring posts that included banned words such as "bitcoin" 
or "net neutrality"\footnote{\url{http://www.bbc.com/news/technology-27100773}}.

Further, centralized management of user data leads to concerns of personal information being 
sold or intruded upon. In recent years, citizens have learned of government programs in multiple 
countries that crawl the web and archive information off of such websites. In fact, there exist 
multiple public services that given a username, scrape reddit and analyze the user's behavior; even 
showing the topics most participated in and "Fun Guessed Data" such as if the user supports
Occupy Wall Street\footnote{\url{http://www.redditinvestigator.com/}}. Oppressive governments may 
act on such data, tracking individuals. Such a threat can affect a user's willingness to share content 
that is political or controversial. 

An effective, fair and safe Web democracy should $(i)$ present content that is relevant to the user's 
interests and $(ii)$ popular, be $(iii)$ resilience to biased reporting and $(iv)$ to user data 
aggregation. Existing social news sites have provided an effective way to accomplish the first two 
goals. However, their centralized nature raises concerns about content bias and users privacy. 
cliq leverages the anonymity and autonomy that a decentralized systems provides to fulfill these 
additional requirements.
\end{section}

\section{Cliq Design}

\subsection{Content in Cliq}
A fundamental task of any social news site is to store, rank and find relevant content posted to the site. While this is straightforward in a centralized system, cliq must manage content spread across random, interim and potentially malicious nodes.
We organize content through cliq similarly to what is common in most social news sites; each 
post is associated with one or more category tags (e.g. politics, sports, or travel). 
This content is stored in a distributed hash table (DHT), where keys are content tags (i.e. ``poiltics", ``Chicago"), and values are the set of posts related to that tag. A post in Cliq is defined by a url, title, tag(s), and metatdata to determine its popularity. For each tag, a subset of nodes with similar node IDs are collectively responsible for managing its content. These duties include storing, ranking and disseminating links uploaded by users with that tag, as well as monitoring for malicious nodes.

\subsection{Cliq metatags}

Centralized social news site also keep tracks of data in addition to its posts. For example, most sites have a list of IPs of known spammers to block. Any site metadata must also be distributed in cliq. Therefore, Cliq also provides metatags which provide several useful services for Cliq users. These metatags are prefixed with an underscore such as \_blacklist and \_tags. 
Metatags are voted on identically to regular tags and can contain information such as bad nodes through the \_blacklist tag and the most interesting Cliq tags through \_tags. 
These metatags can be extremely useful; for example, the most popular blacklisted IP addresses are those which you would most want to avoid communicating with. 

\section{Cliq Implementation}

The Cliq implementation enforces the following design goals: 

\begin{itemize}
\item Show content that is relevent to user interests
\item Show content that is popular
\item Prevent spamming (such that it would violate either i or ii)
\item Enforce basic protections of user anonymity
\end{itemize}

The cliq interface is based upon the following primitives: get, vote, upload, and replicate. 
These primitives enforce the goals above allowing cliq users to view content with the same ease of use of a centralized system with protections additional protections available through a decentralized system. 

\subsection{Connecting to cliq}

Connecting to cliq, because of its P2P design, simply requires connecting to a single peer that is already part of cliq. 
In order to connect, a node simply needs to generate a Node ID which is a random 160 bit hash. 
After joining the network, a node will learn about other nodes and cliq content by both making and responding to peer requests. 
These features of cliq are an extension of the Kademlia architecture.

\subsection{Storing content in cliq}
\label{subsec:upload}

Each node in cliq has a Node ID which is a unique hash used to identify that particular node. Node IDs are dependent on the users' IP address and port number. This prevents a node's ability to manipulate its location in the network.   
Content in cliq is stored by tag on nodes with Node ID closest to Tag ID specified by prefix matching where the Tag ID is simply a hash of that particular tag's name. 
This scheme is used to both facilitate request routing as well as distribute load through the network. 
Content is initially stored in cliq with a call to the upload primitive, specifying the tag to be associated with the content, a title to show with the content, and the content url. 
This information is then stored at the node with Node ID closest to the Tag ID by prefix. 
If the tag already exists, the new item is merged in with the existing tag and its popularity is increased as specified in \ref{subsec:popularity}. 
Otherwise, the new tag is created containing the new item. 
Since posts can be associated with multiple tags, a single upload may cause multiple stores on different nodes in the cliq. 
This makes posts easily retrievable by tag name as well as distributing load by tag throughout the network. 
As the post is seen and voted on, additional metadata is stored in the post such as the IP addresses of nodes that have voted on it as well as the IP addresses of nodes that have received that post in a get. 
The reasons for this metadata should become clear from \ref{sec:spam}. 

\subsection{Viewing cliq content}

The primitive for viewing content is a get. 
A get specifies a tag name and returns popular content associated with that tag. 
The request will be routed through the cliq network based on the Tag ID. 
A Node ID with prefix close to that Tag ID will return posts associated with that tag, if the tag exists. 
The posts that are returned will be a subset of the most popular posts associated with that tag. 
Unless a tag is only controlled by a single node, each node storing content for a particular tag is not necessarily aware of all posts associated with that tag. 
The number of nodes storing each tag is explained in \ref{subsec:replication}. 

\subsection{Determining Popularity}
\label{subsec:popularity}

Popularity is determined by three factors: uploads, up votes, and down votes. 
The formula for determining popularity based on these factors was based on calculations of Reddit data and is explained in \ref{redditshit}.
An upload, as discussed in \ref{subsec:upload}, will either create a new post or will add to the value of an existing post's popularity. 
The vote primitive requries a tag name, content url, and Node ID on which the vote is made. 
Note that a tag name and content url identify a unique post, as this speicifies the criteria for whether or not a new post is created on an upload. 
A voting node will necessarily know of a Node ID to vote on, as in order to find the post it must of have performed a get. 
A high enough popularity for a post on a particular node will trigger a replication, specified in \ref{subsec:replication}. 

\subsection{Replicating content}
\label{subsec:replication}

Tags containing popular content are replicated on multiple nodes close to the Tag ID. 
This helps to distribute the load of gets and votes to a larger subset of nodes such that a single node in the cliq does not have to handle all requests. 
Additionally, this replication provides resilience to churn of users. 
Replication occurs after the popularity of a tag exceeds the replication threshold. In this case, the tag's most popular posts are replicated onto nearby nodes. 
To prevent malicious nodes from replicating arbitrarily on other nodes in the network, a node must send a replication warning to a node it intends to replicate on. 
After receiving a replication warning, a node waits to see votes from a fraction of nodes before accepting the replication request. 
These votes must be redirected from the node wishing to replicate to the node that received the replication warning. 
This verifies that the tag is indeed popular and deserves to be replicated further throughout the cliq network. 
This fraction of the replication tolerance needed to be seen by the node being replicated on is specified by analysis of Reddit and is explained in \ref{subsec:redditshit}. 
The tag is replicated with each post restarting in popularity, with the assumption that popular posts will continue to be popular. 
More discussion of these security measures is in \ref{sec:spam}. 

\section{Spamming Prevention}
\label{sec:spam}
Without registered users or the vantage point of a centralized system, we cannot use traditional methods of detecting spammers. For example, as in Reddit, we cannot ban known spammers because Cliq has no concept of users.  Instead, we take advantage of the fact that in the Kademlia P2P model, a node will be routed to a tag along the same nodes each time. Nodes along this path can collectively enforce good behavior from the users. Several features were added to cliq specifically to avoid a single user or small group of users from controlling the content available in the cliq network. Because the case when a majority of users collude to make items seen by the network is the intended use of Cliq, we look only at the cases of minorities without loss of generality. 

\subsection{In the case of a single malicious node.}
\label{subsec:spamsingle}

Because there is no concept of user accounts in Cliq, an obvious attack vector is a single node that attempts to make a post popular by voting an arbitrary number of times. 
Cliq prevents this by storing the IP address of each node that votes on a particular post and then preventing a node from that IP address from voting again. 
While in general Cliq attempts to store as little data relating users to data, we belive user privacy is perserved for two reasons.
First, any particular node only sees a subset of the votes on a particular post if that post has been replicated and in general can see a small subset of votes in the network. 
And second, we believe protections for IP addresses such as Tor and Psuedonyms can be suitable in addition to using Cliq. 
As well, Cliq prevents lowering popularity arbitrarily through multiple views by keeping a similar list of IP addresses of nodes that receive posts in a get request. 
Another simple attack vector in Cliq is respnding to a get request in Cliq with an arbitrary response. 
While a completely distributed model cannot protect entirely against this type of attack, replication of data impedes the effectiveness of this type of attack. A user can choose from multiple nodes with the desired content. As users vote on content provided by a node, the votes are factored into a weighted model for choosing which node to ask for content. cliq also has the capability to blacklist IPs or node IDs so that they will not be asked for data. 
 
Another attack vector, the simplest way to make content appear popular in the network is by attempting to replicate it arbitrarily. 
Cliq protects against this by requiring during replication that the node attempting to replicate redirect a certain number of up votes to the nodes it wishes to replicate on. 
This number is explained with the analysis of Reddit data in \ref{subsec:redditshit}. 

%%Further, the distribution of which tags and and verification of node IDs ensures random %%distribution of tags a node is responsible for. in multiple node section?

\subsection{In the case of a subset of malicious nodes (Sybil Attack).}
\label{subsec:spammulti}

As with a single node, with no concept of users Cliq must determine popularity such that a small subset of users cannot have votes more impactful than a large number of users. 
The simplest attack vector is simply having a subset of colluding nodes all vote on a post. 
While this will impact the popularity, as it should, it will not do so irrecoverably since popularity is determined as a ratio of up votes and views. 
The inverse case is lowering popularity through multiple views and down votes on a post and is simiilarly handled. 
Since uploads also impact the popularity of a post, this is another attack vector for determining popularity. 
However, Cliq caps the impact of popularity early, with only the first 10 uploads having any impact. 
This should have no legitimite user impact as a node actually attempting to upload content expects that it does not yet exist in Cliq. 
If it has already received 10 uploads, clearly it exists in the network and does not receive an additional popularity boost. 
As discussed in \ref{subsec:spamsingle}, an attack vector exists in replication and a node should not be able to replicate data arbitrarily in the Cliq network. 
Popularity is not perserved during replication. 
If it were, the number of nodes needed to cause a replication would be able to control popularity in the Cliq network. 
This is because a node can choose to replicate on any node in the network with a sufficient number of votes, and can set the popularity arbitrarily high. 
We do not believe that resetting the popularity upon replication negatively impacts the usability of the system as a replicated post will be seen as a new post and therefore will be high ranked initially. 
If the node was popular on the original node, it should continue to be popular on the node it was replicated on since preferences in the Cliq network are assumed to be uniformly distributed through random selection of Node IDs. 
Because popularity is determined exclusively by uploads, votes, views, and upload time, we believe these protections are sufficient. 

\begin{section}{Analysis}

\subsection{Inducing replication}
\label{subsec:inducereplication}


\subsection{Study of reddit}
\label{subsec:redditstudy}

In order to inform the design and analysis of cliq, we perform an analysis of the popular social news site, reddit. We choose reddit because of its popularity (ranked 61 globally on Alexa) and very active user base. Reddit is organized into user-created, topic-based ``subreddit" communities. Subreddits serve a similar function that tags do in cliq. We collect data from over 300,000 most popular subreddits. We look at the distribution of the sizes of the most popular subreddits. We find that only 212 have over 100,000 subscribers. 12 subreddits have over 1,000,000. We therefore expect to perform dynamic load balancing in cliq 
to account for a large variance in tag activity.

\subsection{Load balancing}
\label{subsec:loadbalancing}
TODO



\end{section}

\begin{section}{Conclusion}
  TODO
\end{section}

\bibliographystyle{abbrv}
\bibliography{cliq}

\end{document}

First, a single or small subset of nodes should not be able to cause certain posts to become more (or less) popular than a post that is more widely upvoted.
As such, votes on a particular post within a tag are restricted to one per IP address by the nodes that control them. 
Therefore, it takes a majority of different nodes within the network to cause a post to become popular. 
We do not believe tracking of IP addresses that have voted on a particular post is harmful to the protection of user privacy. 
Nodes controlling a tag will only see a subset of votes if that tag has been replicated. 
In addition, these types of protections are offered by other services that can be used along with cliq, such as Tor. 
As well, when a tag is replicated the popularity of its posts reset as if the posts were recently uploaded. 
Therefore, even if a small subset of nodes are able to cause a tag to replicate, the maximum popularity it will see is the number of nodes that have voted on it. 
This successfully prevents a small subset of nodes from having a disproportionate impact on the content seen by other cliq users. 

