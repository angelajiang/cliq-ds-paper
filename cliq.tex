\documentclass{sig-alternate}
\begin{document}
\nocite{*}

\conferenceinfo{DSCE}{'14 Evanston, Illinois USA}

\title{Cliq: Something Something}
% \subtitle{EECS 395/495: Distributed Systems in Challenging Environments}

\numberofauthors{3}

\author{
\alignauthor Angela Jiang\\
  \affaddr{Northwestern University}\\
  \email{\small angelajiang2014@u.northwestern.edu}
\alignauthor Andrew Kahn\\
  \affaddr{Northwestern University}\\
  \email{\small andrewkahn2015@u.northwestern.edu}
\and
\alignauthor Max New\\
  \affaddr{Northwestern University}\\
  \email{\small maxnew2014@u.northwestern.edu}
}

\date{9 June 2014}

\maketitle
\begin{abstract}
  TODO
\end{abstract}
\begin{section}{Introduction}
  TODO
\end{section}

\section{Cliq Design}

Cliq stores content using a DHT. 
Cliq is built on Kademlia. 
Cliq aggregates content based on popularity. 
Cliq stores urls that users may be interested in. 
People's votes determine which content people see. 

\section{Cliq Implementation}

The Cliq design enforces the following goals: 

\begin{itemize}
\item Show content that is relevent to user interests
\item Show content that is popular
\item Prevent spamming (such that it would violate either i or ii)
\item Enforce basic protections of user anonymity
\end{itemize}

The cliq interface is based upon the following primitives: get, vote, upload, and replicate. 
These primitives enforce the goals above allowing cliq users to view content with the same ease of use of a centralized system with protections additional protections available through a decentralized system. 

\subsection{Connecting to cliq}

Connecting to cliq, because of its P2P design, simply requires connecting to a single peer that is already part of cliq. 
After joining the network, a node will learn about other nodes and cliq content by both making and responding to peer requests. 
These features of cliq are an extension of the Kademlia architecture.

\subsection{Storing content in cliq}
\label{subsec:upload}

Each node in cliq has a Node ID which is a unique hash used to identify that particular node. 
Content in cliq is stored by tag on nodes with Node ID closest to Tag ID specified by prefix matching where the Tag ID is simply a hash of that particular tag's name. 
This scheme is used to both facilitate request routing as well as distribute load through the network. 
Content is initially stored in cliq with a call to the upload primitive, specifying the tag to be associated with the content, a title to show with the content, and the content url. 
This information is then stored at the node with Node ID closest to the Tag ID by prefix. 
If the tag already exists, the new item is merged in with the existing tag and its popularity is increased as specified in \ref{subsec:popularity}. 
Otherwise, the new tag is created containing the new item. 
Since posts can be associated with multiple tags, a single upload may cause multiple stores on different nodes in the cliq. 
This makes posts easily retrievable by tag name as well as distributing load by tag throughout the network. 

\subsection{Viewing cliq content}

The primitive for viewing content is a get. 
A get specifies a tag name and returns popular content associated with that tag. 
The request will be routed through the cliq network based on the Tag ID. 
A Node ID with prefix close to that Tag ID will return posts associated with that tag, if the tag exists. 
The posts that are returned will be a subset of the most popular posts associated with that tag. 
Unless a tag is only controlled by a single node, each node storing content for a particular tag is not necessarily aware of all posts associated with that tag. 
The number of nodes storing each tag is explained in \ref{subsec:replication}. 

\subsection{Determining Popularity}
\label{subsec:popularity}

Popularity is determined by three factors: uploads, up votes, and down votes. 
The formula for determining popularity based on these factors is <FORMULA (min(uploads, 10) + ups - downs)>.
An upload, as discussed in \ref{subsec:upload}, will either create a new post or will add to the value of an existing post's popularity. 
The vote primitive requries a tag name, content url, and Node ID on which the vote is made. 
Note that a tag name and content url identify a unique post, as this speicifies the criteria for whether or not a new post is created on an upload. 
A voting node will necessarily know of a Node ID to vote on, as in order to find the post it must of have performed a get. 
A high enough popularity for a post on a particular node will trigger a replication, specified in \ref{subsec:replication}. 

\subsection{Replicating content}
\label{subsec:replication}

Tags containing popular content are replicated on multiple nodes close to the Tag ID. 
This helps to distribute the load of gets and votes to a larger subset of nodes such that a single node in the cliq does not have to handle all requests. 
Additionally, this replication allows for popular content to stay in the network in the case of nodes disconnecting. 
Replication occurs after the popularity of a post in a particular tag exceeds the replication tolerance, a constant that determines the number of votes needed to replicate. 
To prevent malicious nodes from replicating arbitrarily on other nodes in the network, a node must send a replication warning to a node it intends to replicate on. 
After receiving a replication warning, a node waits to see votes from a fraction of nodes before accepting the replication request. 
These votes must be redirected from the node wishing to replicate to the node that received the replication warning. 
This verifies that the tag is indeed popular and deserves to be replicated further throughout the cliq network. 
This fraction of the replication tolerance needed to be seen by the node being replicated on is specified as <???>. 
The tag is replicated with each post restarting in popularity, with the assumption that popular posts will continue to be popular. 
More discussion of these security measures is in \ref{subsec:spam}. 

\subsection{Spamming Prevention}
\label{subsec:spam}

Several features were added to cliq specifically to avoid a single user or small group of users from controlling the content available in the cliq network.
First, a single or small subset of nodes should not be able to cause certain posts to become more (or less) popular than a post that is more widely upvoted.
As such, votes on a particular post within a tag are restricted to one per IP address by the nodes that control them. 
Therefore, it takes a majority of different nodes within the network to cause a post to become popular. 
We do not believe tracking of IP addresses that have voted on a particular post is harmful to the protection of user privacy. 
Nodes controlling a tag will only see a subset of votes if that tag has been replicated. 
In addition, these types of protections are offered by other services that can be used along with cliq, such as Tor. 
As well, when a tag is replicated the popularity of its posts reset as if the posts were recently uploaded. 
Therefore, even if a small subset of nodes are able to cause a tag to replicate, the maximum popularity it will see is the number of nodes that have voted on it. 
This successfully prevents a small subset of nodes from having a disproportionate impact on the content seen by other cliq users. 

\begin{section}{Conclusion}
  TODO
\end{section}

\bibliographystyle{abbrv}
\bibliography{cliq}

\end{document}
